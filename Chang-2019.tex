\documentclass{beamer}
\usepackage[no-math]{fontspec}
\usepackage{xeCJK}
\setCJKmainfont{Source Han Sans TW}
\hypersetup{colorlinks,linkcolor=}

\usetheme{CambridgeUS}
\title[Prophylaxis in open fracture]{
    Antibiotic prophylaxis in the management of open fractures:
    A systematic survey of current practice and recommendations
}
\subtitle{Yaping Chang \textit{et al}, \textit{J Bone Joint Surg Rev}, 2019}
\author[Chen-Pang He]{何震邦 (Chen-Pang He), Intern}
\date{March 13, 2019}
\institute[CGH]{Cathay General Hospital}

\newcommand*{\solo}[1]{\centering\includegraphics[width=\textwidth, height=0.8\textheight, keepaspectratio]{#1}}

\begin{document}
\maketitle

\section{Introduction}
\begin{frame}{Background}
    \begin{itemize}
        \item Evidence with regard to antibiotic prophylaxis for patients with
              open fractures of the extremities is limited.
        \item A systematic survey addressing current practice and
              recommendations was conducted.
    \end{itemize}
\end{frame}

\begin{frame}{Systematic survey of current practice and recommendations}
    \begin{itemize}
        \item The current article is thus not intended as a systematic review
              of the evidence-based use of antibiotics.
        \item Rather, it is intended as a systematic survey of reports of
              surgeons' practice in the use of prophylactic antibiotics and a
              complementary systematic survey of the available expert
              recommendations.
    \end{itemize}
\end{frame}

\section{Methods}
\subsection{Criteria}
\begin{frame}{Eligibility criteria}
    \begin{itemize}
        \item Published from 2007-01-01 to 2017-06-30
        \item Addressing either or both of 2 questions:
            \begin{itemize}
                \item What regimens (drug, dose, route of administration, start
                      time, and duration) are clinicians using as prophylaxis?
                \item What regimens are recommended in publications?
            \end{itemize}
    \end{itemize}
\end{frame}

\begin{frame}{Eligible studies}
    \begin{itemize}
        \item Multicenter or single-center RCTs, cohort studies, case-control
              studies, and single-arm studies
            \begin{itemize}
                \item $\ge 80\%$ of patients enrolled after 2006-12-31
            \end{itemize}
        \item Review articles
        \item Surveys of surgeons
        \item Orthopedic textbooks
        \item Clinical practice guidelines
        \item Institutional protocols provided online from trauma centers
    \end{itemize}
\end{frame}

\begin{frame}{Exclusion criteria}
    \begin{itemize}
        \item Publications addressing the use of antibiotics in patients with
              known infections or HIV/AIDS.
        \item Publications restricted to pediatric injuries.
    \end{itemize}
\end{frame}

\section{Data sources and search strategy}
\begin{frame}{Databases}
    \begin{itemize}
        \item Embase
        \item MEDLINE
        \item Cumulative Index to Nursing and Allied Health Literature (CINAHL)
        \item Cochrane Central Registry of Controlled Trials (CENTRAL)
        \item Cochrane Database of Systematic Reviews
    \end{itemize}
\end{frame}

\begin{frame}{Keywords}
    \begin{itemize}
        \item Antibiotics
        \item Antimicrobial
        \item Antibiotic prophylaxis
        \item Open fracture
        \item Compound fracture
        \item Gustilo-Anderson type
        \item Fracture fixation
        \item Nonunion
        \item Infection and the names of specific antibiotics
    \end{itemize}
\end{frame}

\begin{frame}{Textbooks}
    \begin{description}
        \item[Database] WorldCat
        \item[Keywords]
            \begin{itemize}
                \item Orthopedic surgery/operation
                \item Open fracture
                \item Antibiotic prophylaxis
                \item Infection prevention
            \end{itemize}
    \end{description}
\end{frame}

\begin{frame}{Guidelines and protocols}
    Consulted Canadian surgeons and their international orthopaedic colleagues
    about sources of clinical practice guidelines and Internet-published
    protocols addressing the management of open fractures.
\end{frame}

\begin{frame}{Keywords for online protocols}
    Open-ended Google search with
    \begin{itemize}
        \item Trauma
        \item Injury
        \item Open fracture
        \item Antibiotic
        \item Guideline or protocol
    \end{itemize}
\end{frame}

\begin{frame}{Sources for online protocols}
    \begin{itemize}
        \item Agency for Healthcare Research and Quality
        \item American Academy of Orthopaedic Surgeons
        \item Orthopaedic Trauma Association
        \item Scottish Intercollegiate Guidelines Network (SIGN)
        \item National Institute for Health and Care Excellence (NICE)
        \item British Orthopaedic Association
        \item East Practice Management Guidelines Work Group
        \item Medscape
        \item SurgWiki
        \item Cambridge Orthopaedics
        \item OrthoBullets
    \end{itemize}
\end{frame}

\section{Results}
\begin{frame}
    \solo{F1.jpg}
\end{frame}

\begin{frame}
    \solo{T1.pdf}
\end{frame}

\begin{frame}
    \solo{T2.pdf}
\end{frame}

\begin{frame}
    \solo{T3.pdf}
\end{frame}

\begin{frame}
    \solo{T4.pdf}
\end{frame}

\begin{frame}
    \solo{T5.pdf}
\end{frame}

\begin{frame}
    \solo{T6.pdf}
\end{frame}

\begin{frame}
    \solo{E1.pdf}
\end{frame}

\begin{frame}
    \solo{E2.pdf}
\end{frame}

\begin{frame}
    \solo{E3.pdf}
\end{frame}

\section{Discussion}
\begin{frame}{Discussion}
    A comprehensive overview of the actual practice and recommendations for
    primary antimicrobial prophylaxis in patients with open extremity fractures
    published from 2007 to 2017 was provided.
\end{frame}

\begin{frame}{Clinical practice}
    Clinicians almost always used broad-spectrum antibiotics rather than
    antibiotics with only reliable gram-positive coverage, regardless of injury
    severity.
\end{frame}

\begin{frame}{Recommendations}
    \begin{itemize}
        \item When authors made recommendations for Gustilo type-I or II open
              fractures as a group, a majority recommended restricting
              antibiotic use to agents with exclusively gram-positive coverage.
        \item When authors grouped Gustilo type-II and III fractures together,
              over 90\% suggested using broad coverage with reliable activity
              against both gram-positive and gram-negative organisms.
    \end{itemize}
\end{frame}

\begin{frame}{Strengths of this study}
    \begin{itemize}
        \item Explicit eligibility criteria
        \item Comprehensive search
        \item Review articles including many traditional and `gray' literature
        \item Clinical practice guidelines in all languages
        \item Duplicate assessment of eligibility
    \end{itemize}
\end{frame}

\begin{frame}{Limitations}
    \begin{itemize}
        \item Deficiencies in reporting in the eligible articles
        \item Most included studies were narrative reviews that made practice
              recommendations.
        \item Recommendations did not provide consistent, comprehensive, and
              detailed descriptions in standard formats that included drug,
              dose, route of administration, start time, and duration.
        \item Details were even more likely to be absent in the reports of
              experience in clinical practice, observational studies, and RCTs.
    \end{itemize}
\end{frame}

\begin{frame}{Conclusions}
    \begin{itemize}
        \item Several key questions remain unresolved with regard to primary
              prophylaxis with antibiotics in patients with open fractures.
        \item Whether there is any benefit of broad compared with targeted
              antimicrobial coverage or of coverage for specific pathogens like
              MRSA or Pseudomonas remains uncertain.
        \item The resolution of these issues, in particular, the degree of
              coverage required, will require well-designed RCTs.
    \end{itemize}
\end{frame}
\end{document}
